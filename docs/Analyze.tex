% Options for packages loaded elsewhere
\PassOptionsToPackage{unicode}{hyperref}
\PassOptionsToPackage{hyphens}{url}
\PassOptionsToPackage{dvipsnames,svgnames,x11names}{xcolor}
%
\documentclass[
  letterpaper,
  DIV=11,
  numbers=noendperiod]{scrartcl}

\usepackage{amsmath,amssymb}
\usepackage{iftex}
\ifPDFTeX
  \usepackage[T1]{fontenc}
  \usepackage[utf8]{inputenc}
  \usepackage{textcomp} % provide euro and other symbols
\else % if luatex or xetex
  \usepackage{unicode-math}
  \defaultfontfeatures{Scale=MatchLowercase}
  \defaultfontfeatures[\rmfamily]{Ligatures=TeX,Scale=1}
\fi
\usepackage{lmodern}
\ifPDFTeX\else  
    % xetex/luatex font selection
\fi
% Use upquote if available, for straight quotes in verbatim environments
\IfFileExists{upquote.sty}{\usepackage{upquote}}{}
\IfFileExists{microtype.sty}{% use microtype if available
  \usepackage[]{microtype}
  \UseMicrotypeSet[protrusion]{basicmath} % disable protrusion for tt fonts
}{}
\makeatletter
\@ifundefined{KOMAClassName}{% if non-KOMA class
  \IfFileExists{parskip.sty}{%
    \usepackage{parskip}
  }{% else
    \setlength{\parindent}{0pt}
    \setlength{\parskip}{6pt plus 2pt minus 1pt}}
}{% if KOMA class
  \KOMAoptions{parskip=half}}
\makeatother
\usepackage{xcolor}
\setlength{\emergencystretch}{3em} % prevent overfull lines
\setcounter{secnumdepth}{-\maxdimen} % remove section numbering
% Make \paragraph and \subparagraph free-standing
\makeatletter
\ifx\paragraph\undefined\else
  \let\oldparagraph\paragraph
  \renewcommand{\paragraph}{
    \@ifstar
      \xxxParagraphStar
      \xxxParagraphNoStar
  }
  \newcommand{\xxxParagraphStar}[1]{\oldparagraph*{#1}\mbox{}}
  \newcommand{\xxxParagraphNoStar}[1]{\oldparagraph{#1}\mbox{}}
\fi
\ifx\subparagraph\undefined\else
  \let\oldsubparagraph\subparagraph
  \renewcommand{\subparagraph}{
    \@ifstar
      \xxxSubParagraphStar
      \xxxSubParagraphNoStar
  }
  \newcommand{\xxxSubParagraphStar}[1]{\oldsubparagraph*{#1}\mbox{}}
  \newcommand{\xxxSubParagraphNoStar}[1]{\oldsubparagraph{#1}\mbox{}}
\fi
\makeatother


\providecommand{\tightlist}{%
  \setlength{\itemsep}{0pt}\setlength{\parskip}{0pt}}\usepackage{longtable,booktabs,array}
\usepackage{multirow}
\usepackage{calc} % for calculating minipage widths
% Correct order of tables after \paragraph or \subparagraph
\usepackage{etoolbox}
\makeatletter
\patchcmd\longtable{\par}{\if@noskipsec\mbox{}\fi\par}{}{}
\makeatother
% Allow footnotes in longtable head/foot
\IfFileExists{footnotehyper.sty}{\usepackage{footnotehyper}}{\usepackage{footnote}}
\makesavenoteenv{longtable}
\usepackage{graphicx}
\makeatletter
\def\maxwidth{\ifdim\Gin@nat@width>\linewidth\linewidth\else\Gin@nat@width\fi}
\def\maxheight{\ifdim\Gin@nat@height>\textheight\textheight\else\Gin@nat@height\fi}
\makeatother
% Scale images if necessary, so that they will not overflow the page
% margins by default, and it is still possible to overwrite the defaults
% using explicit options in \includegraphics[width, height, ...]{}
\setkeys{Gin}{width=\maxwidth,height=\maxheight,keepaspectratio}
% Set default figure placement to htbp
\makeatletter
\def\fps@figure{htbp}
\makeatother

\KOMAoption{captions}{tableheading}
\makeatletter
\@ifpackageloaded{caption}{}{\usepackage{caption}}
\AtBeginDocument{%
\ifdefined\contentsname
  \renewcommand*\contentsname{Table of contents}
\else
  \newcommand\contentsname{Table of contents}
\fi
\ifdefined\listfigurename
  \renewcommand*\listfigurename{List of Figures}
\else
  \newcommand\listfigurename{List of Figures}
\fi
\ifdefined\listtablename
  \renewcommand*\listtablename{List of Tables}
\else
  \newcommand\listtablename{List of Tables}
\fi
\ifdefined\figurename
  \renewcommand*\figurename{Figure}
\else
  \newcommand\figurename{Figure}
\fi
\ifdefined\tablename
  \renewcommand*\tablename{Table}
\else
  \newcommand\tablename{Table}
\fi
}
\@ifpackageloaded{float}{}{\usepackage{float}}
\floatstyle{ruled}
\@ifundefined{c@chapter}{\newfloat{codelisting}{h}{lop}}{\newfloat{codelisting}{h}{lop}[chapter]}
\floatname{codelisting}{Listing}
\newcommand*\listoflistings{\listof{codelisting}{List of Listings}}
\makeatother
\makeatletter
\makeatother
\makeatletter
\@ifpackageloaded{caption}{}{\usepackage{caption}}
\@ifpackageloaded{subcaption}{}{\usepackage{subcaption}}
\makeatother

\ifLuaTeX
  \usepackage{selnolig}  % disable illegal ligatures
\fi
\usepackage{bookmark}

\IfFileExists{xurl.sty}{\usepackage{xurl}}{} % add URL line breaks if available
\urlstyle{same} % disable monospaced font for URLs
\hypersetup{
  pdftitle={Software \& digital practices at CREST},
  pdfauthor={Émilien Schultz - Alexis Guyot - Claire Ecotière - Philippe Pinczon du Sel},
  colorlinks=true,
  linkcolor={blue},
  filecolor={Maroon},
  citecolor={Blue},
  urlcolor={Blue},
  pdfcreator={LaTeX via pandoc}}


\title{Software \& digital practices at CREST}
\author{Émilien Schultz - Alexis Guyot - Claire Ecotière - Philippe
Pinczon du Sel}
\date{2024-10-03}

\begin{document}
\maketitle


\emph{The survey was conducted between July 2024 and september 2024
among CREST researchers on their digital practices. This summary
higlights the main results.}

A total of \textbf{89} respondants answered the survey (\textbf{81}
completed it totally, with a mean duration of \textbf{14} minutes ; only
some question were mandatory so the total number of respondants can
vary).

\subsection{Respondents profile}\label{respondents-profile}

Respondants are mainly ENSAE researchers (\textbf{92\%}), mostly
faculty/phd student, with a high proportion of economists
(\textbf{61\%})

\begin{longtable}[]{@{}llll@{}}
\toprule\noalign{}
& & N & \% \\
\midrule\noalign{}
\endhead
\bottomrule\noalign{}
\endlastfoot
\multirow{3}{=}{Organization} & ENSAE & 82.0 & 92.1 \\
& ENSAI & 7.0 & 7.9 \\
& Total & 89.0 & 100.0 \\
\multirow{7}{=}{Field} & Computer science & 1.0 & 1.1 \\
& Economics & 54.0 & 60.7 \\
& Finance & 5.0 & 5.6 \\
& Other & 3.0 & 3.4 \\
& Sociology & 8.0 & 9.0 \\
& Statistics & 18.0 & 20.2 \\
& Total & 89.0 & 100.0 \\
\multirow{6}{=}{Status} & Faculty & 48.0 & 53.9 \\
& Other & 2.0 & 2.2 \\
& Phd student & 30.0 & 33.7 \\
& Postdoctoral researcher & 8.0 & 9.0 \\
& Research assistant & 1.0 & 1.1 \\
& Total & 89.0 & 99.9 \\
\multirow{5}{=}{Seniority} & Between 1 and 5 years & 42.0 & 47.2 \\
& Between 5 and 10 years & 8.0 & 9.0 \\
& Less than 1 year & 19.0 & 21.3 \\
& More than 10 years & 20.0 & 22.5 \\
& Total & 89.0 & 100.0 \\
\end{longtable}

\subsection{Digital practices}\label{digital-practices}

Almost all respondants (\textbf{87\%}) are involved in digital
processing of data (including simulations) : \textbf{99\%} are using
numeric data, \textbf{51\%} textual data, \textbf{14\%} images and
\textbf{4\%} audio. More specifically, \textbf{29\%} uses experimental
data.

There is a wide diversity of practices at CREST, both regarding
computing activites or storage. More than half of the respondant uses
dataset around 1 Gb or less (\textbf{52\%}), and only \textbf{7\%}
declared to use a dataset bigger than 100 Gb (\textbf{10\%} reported to
not know).

Overall, \textbf{83\%} reported to have enough computing ressources and
\textbf{77\%} they had enought storage. Nevertheless, a few respondents
reported the limit of available ressources.

\begin{quote}
My research is in computational statistics and machine learning. You
can't seriously compete with other teams in this field without access to
big clusters (100s of CPU cores, GPUs).
\end{quote}

The question of getting enough GPU memory (VRAM) for LLM was mentionned
a few time. Comments were made on the necessity of flexible cloud
storage.

The diversity of practices is visible on the hardware used. For
instance, the distribution for the question \emph{`on which computers do
you perform these data processing tasks/computations ?'} :

\begin{longtable}[]{@{}lllllllllll@{}}
\toprule\noalign{}
& \multicolumn{2}{l}{%
{[}Locally, with my office computer{]}} & \multicolumn{2}{l}{%
{[}Locally, on my laptop (GENES/CREST or personal){]}} &
\multicolumn{2}{l}{%
{[}Locally, on a dedicated computer{]}} & \multicolumn{2}{l}{%
{[}With GENES servers{]}} & \multicolumn{2}{l@{}}{%
{[}With server outside GENES infrastructure{]}} \\
& N & \% & N & \% & N & \% & N & \% & N & \% \\
\midrule\noalign{}
\endhead
\bottomrule\noalign{}
\endlastfoot
No & 48.0 & 57.8 & 26.0 & 31.3 & 79.0 & 95.2 & 53.0 & 63.9 & 64.0 &
77.1 \\
Yes & 35.0 & 42.2 & 57.0 & 68.7 & 4.0 & 4.8 & 30.0 & 36.1 & 19.0 &
22.9 \\
Total & 83.0 & 100.0 & 83.0 & 100.0 & 83.0 & 100.0 & 83.0 & 100.0 & 83.0
& 100.0 \\
\end{longtable}

And for the question \emph{`Where do you currently store your data ?'}

\begin{longtable}[]{@{}lllllllllllllll@{}}
\toprule\noalign{}
& \multicolumn{2}{l}{%
{[}Locally, with my office computer{]}} & \multicolumn{2}{l}{%
{[}Locally, on my laptop (GENES/CREST or personal){]}} &
\multicolumn{2}{l}{%
{[}Locally, on a dedicated computer{]}} & \multicolumn{2}{l}{%
{[}With the ABRA server in GENES{]}} & \multicolumn{2}{l}{%
{[}With other GENES servers (than ABRA){]}} & \multicolumn{2}{l}{%
{[}With the CASD outside GENES{]}} & \multicolumn{2}{l@{}}{%
{[}With other servers outside GENES{]}} \\
& N & \% & N & \% & N & \% & N & \% & N & \% & N & \% & N & \% \\
\midrule\noalign{}
\endhead
\bottomrule\noalign{}
\endlastfoot
No & 55.0 & 66.3 & 30.0 & 36.1 & 80.0 & 96.4 & 78.0 & 94.0 & 68.0 & 81.9
& 65.0 & 78.3 & 58.0 & 69.9 \\
Yes & 28.0 & 33.7 & 53.0 & 63.9 & 3.0 & 3.6 & 5.0 & 6.0 & 15.0 & 18.1 &
18.0 & 21.7 & 25.0 & 30.1 \\
Total & 83.0 & 100.0 & 83.0 & 100.0 & 83.0 & 100.0 & 83.0 & 100.0 & 83.0
& 100.0 & 83.0 & 100.0 & 83.0 & 100.0 \\
\end{longtable}

\subsection{Software practices}\label{software-practices}

Regarding operating systems, the majority (\textbf{70\%}) uses Windows,
on third MacOs (36\%) and only 13.5 uses Linux.

A majority of respondants (58\%) reported to need a desktop computer.

The main softwares deemed to be necessary are Stata, R, Python, Matlab,
Dropbox and Latex.

Only 15\% of the respondants are paying software with their research
funds. Some examples are : databases access, chatGPT, MaxQDA\ldots{}
More (43\%) are paying software with their own pocket money : Dropbox,
Claude/ChatGPT, Zotero storage, Overleaf, Dropbox. To note : some of
those software are available in the laboratory offer.

To the question on the needs, several suggestions were made : chatGPT,
Acrobat Pro, Dropbox/Google drive cloud storage, OCR software, or
Premium Overleaf account

The current programming practices at CREST shows the diversity of
languages, with a dominant of R, Python and Stata.

\begin{longtable}[]{@{}lllllllllllll@{}}
\toprule\noalign{}
& \multicolumn{2}{l}{%
r} & \multicolumn{2}{l}{%
python} & \multicolumn{2}{l}{%
julia} & \multicolumn{2}{l}{%
stata} & \multicolumn{2}{l}{%
matlab} & \multicolumn{2}{l@{}}{%
sas} \\
& N & \% & N & \% & N & \% & N & \% & N & \% & N & \% \\
\midrule\noalign{}
\endhead
\bottomrule\noalign{}
\endlastfoot
No & 30.0 & 36.1 & 25.0 & 30.1 & 72.0 & 86.7 & 39.0 & 47.0 & 52.0 & 62.7
& 73.0 & 88.0 \\
Yes & 53.0 & 63.9 & 58.0 & 69.9 & 11.0 & 13.3 & 44.0 & 53.0 & 31.0 &
37.3 & 10.0 & 12.0 \\
Total & 83.0 & 100.0 & 83.0 & 100.0 & 83.0 & 100.0 & 83.0 & 100.0 & 83.0
& 100.0 & 83.0 & 100.0 \\
\end{longtable}

A lot of software paid from researcher's poket

What are the software needed ? Few demands

TO DEVELOP

A vast majority of respondants (\textbf{74\%}) are using generative
model for their research, one third reported to use it a lot. Three out
of for are using free access solutions.

This use cover a diversity of tasks :

\begin{itemize}
\tightlist
\item
  editing english
\item
  copilot for programming / understanding code
\item
  summarizing articles
\item
  formatting references
\item
  email reformulation
\item
  text annotation for research tasks
\item
  exploring topics
\end{itemize}

\subsection{Reproducibility, open source and
evolutions}\label{reproducibility-open-source-and-evolutions}

The respondents were asked on the issues of reproducibility of
computational results, and a majority considers it important issues for
them.

\begin{longtable}[]{@{}lll@{}}
\toprule\noalign{}
& Effectif & Pourcentage (\%) \\
\midrule\noalign{}
\endhead
\bottomrule\noalign{}
\endlastfoot
These are distant or non-existent issues in your activity & 14.0 &
17.3 \\
These are important issues in your activity & 40.0 & 49.4 \\
You have encountered these issues in your activity & 27.0 & 33.3 \\
Total & 81.0 & 100.0 \\
\end{longtable}

On the question of the adoption ofopen source and free software
solutions, only \textbf{22\%} reported to not be familiar with this
trend, and \textbf{41\%} reported to be actively involved.

Digital practices are constantly evolving, and new solutions are
emerging. An open field allowed to collect insights and comments on
those evolutions. For instance, tools like Dropbox or Google Drive
seemed to have found their place in the research workflow. This
underline the importance of flexible storage.

\begin{quote}
My co-authors at other research institutions (Berkeley, Chicago,
Columbia, Harvard) all use Dropbox and/or Google Drive to store the data
involved in our research projects. To work collaboratively on our
projects, I therefore need to use Dropbox and Google Drive, on which
relatively large amounts of data are stored (about 3 To for my currently
active projects). The free tier plans are not sufficient: I need to pay
for increased storage on both Dropbox and Drive.\nThe current offering
from CREST to use OneDrive is unfortunately not sufficient.
\end{quote}

It appears a need for improvement in accessing existing solutions : SSH
connection to the servers, more information of available ressources,
etc. There is a demand for more training sessions on software and IT
offerings (\textbf{40\%}), for instance parallel computing, Git/Github
good practices, and an overview of available ressources at CREST.

For instance, the development of solutions as Onyxia is mentioned as a
positive point to access shared ressources. For the moment, only
\textbf{22\%} of the respondents used Onyxia once, with \textbf{33\%}
having no idea of its existence.

\begin{quote}
``If there is an infrastructure that supports me, I would love to make
my research data accessible and reproducible. Onyxia is a good tool and
that should be promoted for usage within our laboratory.''
\end{quote}

The necessity to get available GPU for deep learning and LLM practices
is clearly highlighted. Another important point was the need for secured
storage accredited to store sensitive informations.

\begin{quote}
``In terms of architecture, I'm overall satisfied with what we have in
the department. The only need I have that is not currently fulfilled
would be a computing server sufficient secure to store confidential
data. Something in between the existing computing servers and CASD
(which cannot host easily confidential data that I can obtain from third
parties, and is very expensive). It would be amazing to set up this kind
of server.''
\end{quote}

Finaly, the developement of AI assistant creates new practices which
encounter several constraints

\begin{quote}
``We need more flexibility in the use of online service such as AI
(Chatgpt, Claude, \ldots), Deepl, Overleaf, Grammarly, \ldots{} In
particular, it's completely absurd that we can't use our research fund
to pay for the online service.''
\end{quote}




\end{document}
